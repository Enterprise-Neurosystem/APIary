\paragraph{Community Engagement}
Climate events are occurring with increasing rapidity. 
Our effort hopes to break ground with demonstration cases that spur partnerships across public and private sectors, across government agencies, and across divides and gaps between cultural and socioeconomic groups. 
Such partnerships are a requisite element \cite{SDGs} that underlies any national effort to build adaptive and resilient energy and climate strategies and policies that can leverage the latest emerging technologies in AI, machine learning, robotics, and applied mathematics. 
This initiative will lead the path for robust and adaptive precision agriculture and an intelligent equitable energy strategy that at once recognizes the individual interests of all active participants and shareholders while predicting climate and humanitarian events and optimizing dynamic adaptation to accommodate and even capitalize on them.
Initiated under the auspices of an open-source community group called The Enterprise Neurosystem, which includes members from SLAC/Stanford, multiple universities, and private and public companies, our community is coalescing around a very wide-reaching effort for an AI infrastructure that promotes climate resilience via adaptive agricultural practices.
As an initial demonstration of the utility of broad information sharing for model training, the team is working in close collaboration with the Department of Energy's AI and Technology Office (AITO) to develop a secure integration of distributed datasets that intentionally span multiple scientific domains in a federation model that preserves data privacy where needed, openly exposes publicly shared data where available, and creates mutually beneficial AI models that generalize across domains of human knowledge.

\paragraph{The Hive Mind}
Our pursuit of active adaptation to climate events will foster resilient agriculture and energy systems with the aim to bring artificial intelligence centrally into the task of integrating the full range of human environmental monitoring and response.
We have an ultimate goal of overlaying climate-relevant satellite imagery with local agricultural health.
We have begun with a project developed from a grass-roots partnership between the DOE's AITO, SLAC's cross-domain EdgeAI research, and the open-source community of industry partners--the Enterprise Neurosystem.
Our shared vision is to merge together information from multiple modes of environmental monitoring and at vastly different geospatial scales.
For instance, global scale environmental satellite imagery, from meters to kilometers, individually focusing in on crop diversity and regional pollution from refuse \cite{TomasBeaches} and agricultural runoff \cite{NeonicsOnBees} as well as aquifer health \cite{GRACE_CongoBasinWatershed} and surface water concentrations \cite{SurfaceWater}.
While a satellite modality might capture CO$_2$ concentration \cite{GlobalCO2} at the kilometer meter scale, urban and agricultural beehive health could give a finer mesh of overall environmental health.
Correlating the hive health with variations in the numerous environmental factors is just the job of data-hungry machine learning algorithms.

But this project starts a few beehives at a time.
Given the wealth of large-scale satellite imagery available, we have turned our attention to ground-level primary sources, we plan to ask the bees themselves. 
Stanford collaborator Dr. Tadashi Fukami uses precise individual bee micro-biome analysis \cite{Fukami2022} to reconstruct the nectar diversity for a hive.
Our team is currently focused on continuous passive acoustic monitoring of hive health.
Starting small and local, we are quickly growing to partner with a much broader diversity of hive locations across complementary environments such as industrial and local agricultural sites, native tribal lands, urban sites, and international sites where domestication of native honeybee species is the common practice. 
This growth into diverse environments is precisely the benefit of engaging with the US Department of Energy Artificial intelligence and Technology Office.
By working directly with typically under-engaged institutions that represent just the agricultural and urban demographics that are key to project success, AITO is helping foster AI development in the communities that will likely have an outsized positive impact on the future of climate adaptability.

Figure~\ref{fig::photos} shows the first prototype testing of a tiny acoustic microphone, placed inside the beehive, that records the hive humming sound well above the human acoustic regime.
The sensors are designed of easily attainable parts and constrained to require little advanced knowledge for construction.
Our team is designing the sensor for open sharing of the recipe in order to encourage the broader agricultural and environmental-minded communities to directly and personally engage with the effort in community ownership and co-design paradigm.
Given the Red Hat origin of the Enterprise Neurosystem community, this project sticks to a philosophy of interdisciplinary open-source sharing of ideas and methods, including hardware integration and software design.
As a community endeavor, all members will share a sense of ownership of the technology.

\paragraph{Sharing is Caring}
As a seed idea to catalyze the direction for this multi-disciplinary endeavor for climate and energy resilience, we take inspiration from the accomplishments that so-called Transformer \cite{Attention2017} or Foundation \cite{StanfordFoundationPaper} models have gleaned from tremendously diverse training corpora.
The tremendous results in language completion tasks for so called transformer or foundation models like BERT \cite{bert} and GPT \cite{GPT2018} have been achieved precisely owing to the diverse language corpora available to use as training sets.
Unfortunately, the ability of tremendously large models to leverage such large datasets requires massive computational capability, a capability that is reserved for major industrial entities and major computational facilities in DOE.
Innovation, however, comes from non-traditional combinations of human ideas, combinations that shine spotlights into the gaps between research fields.
Similarly, large machine learning (ML) models attain ``generalizability'' by training across increasingly diverse training sets.
Along this vain, even though our project is focused on acoustics for hive health, we plan to leverage compatible format data from fields as far-reaching as data-center acoustics and Tokamak magnetic fusion diagnostics (see Fig.~\ref{fig::spectrograms}).


\paragraph{Sharing is Daring}
The challenge of open data sharing is a topic that the research sector finds unnervingly challenging.
The personal motivations that drive researchers to push the limits of possibility are often the same motivations that discourage openly sharing uniquely produced datasets openly with understandably competitive community members.
This personal motivation of Principle Investigators (PIs) is not unlike the intellectual property (IP) concerns of private sector entities or even economic aspirations of both developed and developing communities and nations.
As noted in the UN Development Programme's sustainable development goals \cite{SDGs}, addressing global climate goals will require equitable partnerships that respect the economic aspirations of participants.
The US research ecosystem, in particular across the Department of Energy, represents a functionally similar challenge for equitable sharing in support of a cohesive and sustainable future for broad adoption of AI practices that will only succeed in a multi-modal, data-rich, and geographically and culturally distributed national computing ecosystem.
Solving a grand challenge and building a far reaching vision, however, always begins with a small exemplar--one hive at a time.



