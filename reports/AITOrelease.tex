
\paragraph{Community Engagement}
Climate events are occurring with increasing rapidity. 
In response to the urgency for action, a community is coalescing around a very wide-reaching effort for an artificial intelligence (AI) infrastructure that promotes climate resilience via adaptive agricultural practices, and provides course correction recommendations that mitigate the human and environmental impact of climate events.
Initiated by Red Hat Inc.~as an open-source community group called Enterprise Neurosystem, this community includes members from SLAC/Stanford, multiple universities, and multiple private and public companies.
We plan to use demonstration cases across use-case domains that spur partnerships across public and private sectors, span government agencies, and cross the divide between cultural and socioeconomic groups.
Such partnerships lie at the foundation of any effective national effort to build adaptive and resilient energy and climate strategies and policies \cite{SDGs} that can leverage the latest emerging technologies in AI, machine learning, robotics, and applied mathematics. 

One such inspirational demonstration for broad information sharing for AI model training consists of a close collaboration with the Department of Energy's Artificial Intelligence and Technology Office (AITO) to develop the secure integration of distributed datasets that intentionally span multiple scientific domains. 
The collaboration aims to lead the path for robust and adaptive precision agriculture and an intelligent equitable energy strategy that at once recognizes individual interests of all active participants and shareholders as well as foreseeing climate and humanitarian events and using that foresight to optimize a dynamic adaptation actions that at least accommodate and at best capitalize on them.

The collaboration between SLAC, Enterprise Neurosystem, and the AITO aims for a federation model for AI that preserves data privacy as needed, openly exposes publicly shared data where available, and creates mutually beneficial AI models that generalize across relevant domains of human knowledge.
Building a cross-domain federation model as a partnership, rather than as a single entity, ensures that this initial demonstration will encourage optimal alignment between federal and private AI infrastructure and interoperability.


\paragraph{The Hive Mind}
Our pursuit of active adaptation to climate events that fosters resilient agriculture and energy systems aims to bring AI centrally into the task of integrating the full breadth of environmental monitoring and response. 
With plans to overlay openly shared CO$_2$ and methane relevant satellite imagery with local agricultural health we are beginning with a project developed as a grass-roots partnership between the DOE's AITO, SLAC's cross-domain EdgeAI research, and the open-source community of industry partners via the Enterprise Neurosystem.
Our shared vision is to merge information from multiple modes of environmental monitoring at vastly different geospatial scales.

\begin{wrapfigure}{l}{.5\linewidth}
	\centerline{
		\includegraphics[trim={0 150 0 50},clip,width=\linewidth]{./localfigs/Composite.jpg}
		}
		\vspace{-1\baselineskip}
	\caption{\label{fig::composite}
		Composite of satellite imaging of various modalities, and regions, based on Refs.~\cite{YaraGlaciers,TomasBeaches,GlobalCO2,SurfaceWater} for a-d respectively.
		}
\end{wrapfigure}

For instance, global scale environmental satellite imagery (Fig.~\ref{fig::composite}), from meters to kilometers, can provide a family of diagnostic overlays.
These overlays will connect with point samples of beehive health on the ground much like the water sample points shown in Fig.~\ref{fig::composite}d adapted from Ref.~\cite{SurfaceWater}.
Given the wealth of large-scale satellite imagery available, we have turned our attention to the ground-level primary source data generation; we plan to ask the bees themselves.

We will correlate beehive acoustic measurements with overlays that individually focus in on crop diversity and regional pollution from refuse \cite{TomasBeaches}, agricultural runoff \cite{NeonicsOnBees} as well as aquifer \cite{GRACE_CongoBasinWatershed} and glacier health \cite{YaraGlaciers}, and surface water concentrations \cite{SurfaceWater}. 
A satellite modality can capture CO$_2$ \cite{GlobalCO2} and methane \cite{MethaneEmission2022} concentrations at the kilometer scale, e.g.~fine grained chemically but coarse grained spatially.
Our proposal for urban and agricultural beehive health would give a complimentary fine-grained spatial and coarse grained chemical monitor of environmental health.
Correlating hive health with variations in numerous environmental factors is just the job of data-hungry machine learning algorithms. 

Stanford collaborator Dr. Tadashi Fukami uses precise individual bee microbiome analysis \cite{Fukami2022} to reconstruct the nectar diversity for a hive. 
One can imagine point measurements of nectar diversity attributing spatial labels into geospatial images much like the water sample points (dots) in Fig.~\ref{fig::composite}d.
Our team is currently focused on continuous passive acoustic monitoring that would add point measurements of hive health, queen state, and agitation level. 
Starting small and local, a few beehives at a time, we are quickly growing to partner with a much broader diversity of hive locations across complementary environments--industrial and local agricultural sites, native tribal lands, urban sites, and international sites where domestication of native honeybee species is a common practice.

Growth into diverse environments is precisely the benefit of engaging with the AITO.
By working directly with typically under-engaged institutions that represent the agricultural and urban demographics that are key to project success, AITO is helping foster AI development in the communities that will have specific and outsized positive impacts on the future of climate resilience and adaptability.
Given the Red Hat origins of the Enterprise Neurosystem community, this project sticks to a philosophy of interdisciplinary open-source sharing of ideas and methods, including hardware integration and software design. 
As a community endeavor, all members share a sense of ownership of the technology.

\begin{figure}
	\centerline{a.\hspace{.5\linewidth}b.}
	\centerline{
		\includegraphics[clip,height=.7\linewidth]{../figs/ToddZooms.png}
		\includegraphics[trim={-10 -20 -10 -10},clip,height=.7\linewidth]{../figs/IMG_4990.jpg}
		}
	\caption{\label{fig::bees}
		a. Sequential zoom into the in-hive communication. 
		Initial agitated state from recently closed hive, can be identified by high amplitude, broad frequency, humming that quiets after about 3000 time slices.
		The bottom zoom window shows that what may look like noisy features in the spectrogram are actually controlled frequency chirps.
		b. Image of a prototype acoustic sensor with frequency response from 20Hz to 85kHz. 
		}
\end{figure}

Figure~\ref{fig::bees} shows the first prototype testing of a tiny ultrasonic microphone, placed inside the beehive, that records the hive humming at well above human acoustic perception.  
The sensors are designed with easily attainable parts, require little advanced knowledge for construction, and cost less than \$3 US each when complete. 
Our team is also providing the AI software and design elements as open-source, free of charge, to encourage more than just simple adoption in broader agricultural and environmental-minded communities, we explicitly encourage direct engagement with us in order to foster collective solution ownership via a robust co-design paradigm.


%\begin{wrapfigure}{l}{.5\linewidth}
%	\centerline{a.}
%	\centerline{
%		\includegraphics[clip,width=.9\linewidth]{../figs/agitation_filt.jpg}
%		}
%	\centerline{b.\hspace{.3\linewidth}c.\hspace{.3\linewidth}d.}
%	\centerline{
%		\includegraphics[clip,width=.3\linewidth]{../figs/postagitation_filt.jpg}
%		\includegraphics[clip,width=.3\linewidth]{../figs/communicationzoom_filt.jpg}
%		\includegraphics[trim={-25 -50 -25 0},clip,width=.3\linewidth]{../figs/IMG_4990.jpg}
%		}
%	\vspace{-1\baselineskip}
%	\caption{\label{fig::bees}
%		a. Agitated initial state and calming.
%		b. Post-calming communication.
%		c. Calm communication.
%		d. Prototype acoustic sensor with frequency response from 20Hz to 85kHz.
%		}
%\end{wrapfigure}


\paragraph{Sharing is Caring}
As a seed idea to catalyze the direction for this multidisciplinary endeavor for climate and energy resilience, we take inspiration from the accomplishments that so-called Transformer \cite{Attention2017} or Foundation \cite{StanfordFoundationPaper} models have gleaned from tremendously diverse training corpora \cite{bert,GPT2018}.
%The impressive results in language completion tasks for such AI models \cite{bert,GPT2018} have been achieved precisely owing to the diverse language corpora available as training sets. 
Unfortunately, the ability of tremendously large models to leverage vast datasets requires equally massive computational capability--a capability typically reserved for major industrial entities or the largest computational facilities in DOE. 

Innovation, however, comes from non-traditional combinations of human ideas, and these combinations shine a spotlight into the gaps between research fields. 
Similarly, large machine learning (ML) models attain ``generalizability'' by training across increasingly diverse datasets. 
And although our project is focused on acoustics for beehive health, we plan to leverage compatible format data from fields as far-reaching as data-center acoustics and tokamak magnetic fusion diagnostics (see Fig.~\ref{fig::spectrograms}).
We note the similar structure of information in the tokamak spectrogram compared to that of Fig.~\ref{fig::bees}, of course on different time-frequency scales.
This similarity inspires us to represent both beehive and data-center acoustics as we do the tokamak diagnostics in order to augment the diversity of input when jointly training foundation of the tranformer model.

\begin{figure}
	\raggedright{a.}\\
	\centerline{\includegraphics[trim={0 0 0 325},clip,width=\linewidth]{../figs/cerinefigs/ground_truth.png}}
	\raggedright{b.}\\
	\centerline{\includegraphics[trim={0 0 0 325},clip,width=\linewidth]{../figs/cerinefigs/direct_predictions_old_embeddings.png}}
	\caption{\label{fig::spectrograms}
		a. Ground truth spectrogram of ECE signal. b. Forecast of signal by 5ms.}
\end{figure}

\paragraph{Sharing is Daring}
The challenge of open data sharing is a topic that the research sector finds unnervingly challenging. 
The personal motivations that drive researchers to push the limits of possibility are often the same motivations that discourage the open sharing of uniquely produced datasets with understandably competitive community members. 
This personal motivation of principal investigators (PIs) is not unlike the intellectual property (IP) concerns of private sector entities, or the economic aspirations of both developing and developed nations alike.  
As noted in the UN Development Programme's sustainable development goals \cite{SDGs}, addressing global climate challenges will require equitable partnerships that respect the economic aspirations of all participants.

The US research ecosystem, in particular across the Department of Energy, represents a functionally similar challenge for equitable data sharing.
Our national AI ecosystem must encourage inter-lab competition while at the same time optimizing and supporting a convergence of the diverse palette of unique capabilities.
%We must build a cohesive and sustainable future for the broad adoption of AI practices. 
We can only find true success in a multi-modal, data-rich, and geographically and culturally distributed national computing and AI ecosystem. 

Solving a grand challenge and building a far reaching vision, however, always begins with a small exemplar--one beehive at a time.



